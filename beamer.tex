% !TeX encoding = UTF-8
% !TeX spellcheck = de_DE
% !TeX program = lualatex
% !BIB program = biber

\documentclass[ngerman, aspectratio=169]{beamer}
\usepackage{polyglossia} \setdefaultlanguage{german}

\usepackage{fontspec} \setsansfont{Noto Sans} \setmonofont{Noto Mono}
\usepackage{unicode-math}

\usepackage{multicol}

\usetheme{frisbeesportverband}

\title{Ernährung im Ultimate}
\date{\today}
\author[M. Brandt]{Matthias Brandt}
\institute{Deutscher\par Frisbeesport-Verband e.V.}

\begin{document}
\begin{frame}
  \titlepage
\end{frame}

\begin{frame}
  \frametitle{Übersicht}
  \tableofcontents
\end{frame}

\section{Grundlagen}
\begin{frame}
  \frametitle{Nährstoffe}
  \begin{multicols}{2}
    \begin{itemize}
      
    \item Hauptnährstoffe
      \begin{itemize}
      \item Kohlenhydrate
      \item Protein
      \item Fett
      \item Alkohol
      \item Ballaststoffe
      \end{itemize}
      
    \item Wasser
      
      \vfill\null
      \columnbreak
      
    \item Mikronährstoffe
      \begin{itemize}
      \item Fettlösliche Vitamine (E, D, K, A)
      \item Wasserlösliche Vitamine (B, C)
      \end{itemize}
      
    \item Mineralstoffe
      \begin{itemize}
      \item Mengenelemente (Elektrolyte)
      \item Spurenelemente
      \item Ultraspurenelemente
      \item Sekundäre Pflanzenstoffe
      \end{itemize}

    \end{itemize}
    
  \end{multicols}
  
\end{frame}

\section{Kohlenhydrate}
\begin{frame}
  \frametitle{Kohlenhydrate}
  \begin{itemize}
  \item Kohlenhydrate sind wichtigster Energielieferant
  \item Sportler nehmen oft zu wenige Kohlenhydrate auf
  \end{itemize}
\end{frame}

\begin{frame}
  \frametitle{Kohlenhydrate Auswahl}
  \begin{itemize}
  \item Grundbaustein: Monosaccharide (Einfachzucker)
    \begin{itemize}
    \item Glucose, Fructose (Früchte, Honig, kleine Mengen in Pflanzen)
    \end{itemize}
  \item Zwei- und Mehrfachzucker:
    \begin{itemize}
    \item Disaccharide
      \begin{itemize}
      \item Saccharose (Zuckerrübe, Zuckerrohr, Früchte)
      \item Lactose (Milch und Milchprodukte)
      \item Maltose (Abbauprodukt von Stärke)
      \end{itemize}
    \item Oligosaccharide (3-10 Einfachzucker)
      \begin{itemize}
      \item Raffinose (Zuckerrübenmelasse, Honig, Hülsenfrüchte)
      \end{itemize}
    \item Polysaccharide (>10 Einfachzucker)
      \begin{itemize}
      \item pflanzliche Stärke (Stärke, Getreide, Kartoffeln)
      \item tierische Stärke (Artischoken, Chicorée, Zwiebeln, Spargel, Bananen)
      \end{itemize}
    \item Künstliche Zucker
      \begin{itemize}
      \item Dextrose
      \item Glukosesirup
      \end{itemize}




    \end{itemize}

  \end{itemize}
\end{frame}
\begin{frame}[allowframebreaks]
  \frametitle{Glykämischer Index}
  \begin{itemize}
  \item Reaktion des Blutzuckerspiegels auf ein Lebensmittel
  \item Hoher GI $→$ Höherer Blutzuckerspiegel
  \item Höherer Blutzuckerspiegel $→$ Größere Insulinantwort
  \item Größere Insulinantwort $→$ Schnellere Energieabgabe
  \item Beispiele
    \begin{itemize}
    \item Glukose - 100
    \item Weißes Baguette - 95
    \item Banane - 58
    \end{itemize}

  \end{itemize}

  Aber

  \begin{itemize}
  \item Blutzuckerspiegel bei Sportlern steigt langsamer
  \item GI von Mahlzeiten schwer abschätzbar
  \end{itemize}
  
  \framebreak

  Vereinfacht

  \begin{itemize}
  \item Hoher GI für schnellen Energiebedarf (hyperglykämische Lebensmittel)
    \begin{itemize}
    \item Während des Sports oder bei kurzen Regenerationsphasen
    \item Hohe Aufnahme von Glucose und Saccharose vermeiden
    \item Stark verarbeitete Lebensmittel und Süßes
    \end{itemize}

  \item Niedriger GI für normalen Energiebedarf (hypoglykämische Lebensmittel)
    \begin{itemize}
    \item Vor der Belastung
    \item Unverarbeitete Lebensmittel, Vollkornprodukte
    \end{itemize}

  \item Kombination aus beidem für lange Wettkampftage

  \end{itemize}
  
\end{frame}

\begin{frame}
  \frametitle{Bedeutung für Sportler}

  \begin{itemize}
  \item Kohlenhydrate sind wichtiger Energielieferant
  \item Vor dem Sport: Nahrung mit niedrigem GI
  \item Während des Sports: Nahrung mit hohem GI
  \item 
  \end{itemize}
  
\end{frame}


\begin{frame} 
  \frametitle{Blocks, enumerates and a long title} 
  \begin{block}{\centering Theorem}
    There is no largest prime number.
  \end{block} 

  \begin{enumerate} 
  \item<1-| alert@1> Suppose $p$ were the largest prime number. 
  \item<2-> Let $q$ be the product of the first $p$ numbers. 
  \item<3-> Then $q+1$ is not divisible by any of them. 
  \item<4-> But $q + 1$ is greater than $1$, thus divisible by some prime
    number not in the first $p$ numbers.
  \end{enumerate}
\end{frame}

\begin{frame}
  \frametitle{A shorter title}
  \begin{itemize}
  \item one
    \begin{itemize}
    \item one.one
    \end{itemize}
  \item two
  \end{itemize}
\end{frame}

\section{Quellen}
\begin{frame}
  \frametitle{Quellen}
  Quellen

\end{frame}


\end{document}
