% !TeX encoding = UTF-8
% !TeX spellcheck = de_DE
% !TeX program = lualatex
% !BIB program = biber

\documentclass[ngerman, aspectratio=169]{beamer}
\usepackage{polyglossia} \setdefaultlanguage{german}

\usepackage{fontspec} \setsansfont{Noto Sans} \setmonofont{Noto Mono}
\usepackage{unicode-math}

\usetheme{frisbeesportverband}

\title{Ernährung im Sport}
\date{\today}
\author[M. Brandt]{Matthias Brandt}
\institute{Deutscher\par Frisbeesport-Verband e.V.}

\begin{document}
\begin{frame}
  \titlepage
\end{frame}

\begin{frame}
  \frametitle{Übersicht}
  \tableofcontents
\end{frame}

\begin{frame} 
  \frametitle{Blocks, enumerates and a long title} 
  \begin{block}{\centering Theorem}
    There is no largest prime number.
  \end{block} 

  \begin{enumerate} 
  \item<1-| alert@1> Suppose $p$ were the largest prime number. 
  \item<2-> Let $q$ be the product of the first $p$ numbers. 
  \item<3-> Then $q+1$ is not divisible by any of them. 
  \item<4-> But $q + 1$ is greater than $1$, thus divisible by some prime
    number not in the first $p$ numbers.
  \end{enumerate}
\end{frame}

\begin{frame}
  \frametitle{A shorter title}
  \begin{itemize}
  \item one
    \begin{itemize}
    \item one.one
    \end{itemize}
  \item two
  \end{itemize}
\end{frame}

\section{Quellen}
\begin{frame}
  \frametitle{Quellen}
  Quellen
\end{frame}


\end{document}
